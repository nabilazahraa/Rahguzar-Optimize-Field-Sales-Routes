This section explores the techniques and methodologies used for optimizing Permanent Journey Plans (PJP), emphasizing their relevance to modern logistics and distribution systems. It highlights a range of optimization algorithms, including Dynamic Programming, Genetic Algorithms, Particle Swarm Optimization, Ant Colony Optimization, and Mixed-Integer Linear Programming, alongside the role of clustering methods like K-means. These approaches address critical challenges such as minimizing travel distances, adhering to time constraints, balancing workloads, and ensuring scalability in large-scale routing problems. The review establishes a foundation for understanding how these techniques contribute to efficient and effective PJP optimization.

\subsection{Journey Planning for Industries}
In today’s highly competitive marketplace, the speed and efficiency of transportation can vastly influence the success and failure for industries like fast-moving consumer goods, retailers, logistics, and distribution centers \cite{corona2022cumulative}. For these sectors, logistics can
account for over 30\% of distribution costs and contribute 15-40\% to the overall product price \cite{Cui_2012a}. Therefore, businesses need to plan their journey routes with the utmost efficiency to enhance their operational productivity and reduce their costs. The journey plans generated, known as Permanent Journey Plans, or PJPs, have to meet meticulous criteria such as minimizing the distance traveled, making sure time-sensitive plans are met with punctuality, and maximizing the number of places that can be visited on a certain day \cite{Sandhya_Goel_2018}. Traditional approaches to journey planning often rely on manual processes or heuristic methods, making it a tedious process and unable to handle the complexities required by modern logistics systems,  increasing the logistics cost by 10-30\% \cite{Maini_Goel_2017, Alexis_2023}. For developing countries such as Pakistan, the wholesale and retail trade sector contributes to 18\% of the overall GDP, emphasizing on the need to further improve the routing efficiency to support the financial stability and growth of businesses in the retail sector \cite{javed2023undocumented}.

Recent developments in algorithmic approaches, combined with advancements in operations research and computational techniques, have renovated how to assess journey planning problems, and how we can address them \cite{almaatani2014new}. The Vehicle Routing Problem (VRP) is an NP-hard problem widely recognized in journey planning, serving as a benchmark for testing and improving optimization algorithms. These problems are applied in scenarios that require the optimization of Permanent Journey Plans, or PJPs, where the routes and schedules created have to be consistent over extended periods. The created PJPs must account for several constraints such as frequency of visits that ensure locations are serviced at appropriate intervals to meet the customer demands, time window constraints to align the schedules with store operating hours, and distributing workload evenly across the number of available sales force. Moreover, it is vital to balance the rigidity of fixed routes with the dynamic nature of real-world variables, such as changes in traffic, demand fluctuations, and operational constraints.

\subsection{Optimization Techniques in Route Planning
}
% Solving complex, NP-hard problems like Permanent Journey Plan (PJP) optimization requires advanced techniques to balance constraints and objectives efficiently \cite{Alam_Sulaiman_2022}. This is done using different algorithms that focus on improving the performance, efficiency, or effectiveness of a system which would be a time-consuming task when done manually or using brute-force techniques. Several optimization techniques have been adopted for journey planning that can address diverse scenarios such as static routing and performing adjustments to dynamic changes according to varying demands of the user. Commonly used optimization methods include Mixed-Integer Linear Programming (MILP), Dynamic Programming (DP), Genetic Algorithms (GA), Particle Swarm Optimization (PSO), and Ant Colony Optimization (ACO).

% MILP provides exact solutions by modeling objectives and constraints, such as visit frequencies and workload balancing, as linear functions. While computationally intensive for large-scale problems, it is often combined with clustering methods to manage complexity \cite{Laporte2017}. DP decomposes complex problems into smaller subproblems, solving each recursively and storing intermediate results. However, the need to store solutions for all subproblems can lead to exponential growth in problem size, making DP computationally expensive for large-scale problems with many locations or variables. Despite this, DP remains useful for smaller problems or in hybrid approaches to improve solution accuracy \cite{feidiao_yang__2018}.
% GA are well suited for multi-objective optimization problems, iteratively refining a population through selection and crossover schemes, and through mutation \cite{Konak_Coit_Smith_2006}. In the context of route optimization and PJP planning, GAs can be useful to balance objectives such as minimizing distance traveled and meeting the timeframe of the salesforce. PSO , on the other hand, is implemented by mimicking the social behavior of swarms to converge to an optimal solution, classifying it as a metaheuristic approach \cite{Hughes_Goerigk_Dokka_2020}. PSO requires less computing work, making it an efficient algorithm for finding the most optimal solution. Moreover, PSOs can be adapted to provide an optimal cluster containing a set of locations that a certain sales representative would need to visit and can be combined with clustering algorithms such as K-means \cite{inproceedings}. ACO algorithms, based on the foraging behavior of ants, are adept at solving combinatorial optimization problems like the Vehicle Routing Problem (VRP) \cite{articleACO}. Their probabilistic approach to exploring solutions allows them to identify efficient paths while avoiding local optima.

% In addition to traditional optimization algorithms, clustering is an essential pre-processing step in addressing large-scale PJP problems. Clustering involves partitioning a set of locations or territories into subsets, known as clusters, based on proximity or other criteria, which enables better route planning and resource distribution. The goal of clustering is to group locations in a way that minimizes the total travel distance within each group, thereby simplifying the overall route planning process.\\
% For PJP planning, clustering can break down a large, complex problem into smaller, more manageable sub-problems. This segmentation allows route planning algorithms to operate within a cluster independently, ensuring that each sales representative or vehicle has a well-defined region to service. By addressing each cluster separately, the optimization process becomes more computationally feasible, and the resultant solutions are often more practical for real-world applications. This method is particularly effective when combined with algorithms such as GA, PSO, or ACO, which can then be applied to determine the optimal routes within each cluster. 

Solving complex, NP-hard problems like Permanent Journey Plan (PJP) optimization requires advanced techniques to balance constraints and objectives efficiently \cite{Alam_Sulaiman_2022}. Optimization methods such as Mixed-Integer Linear Programming (MILP), Dynamic Programming (DP), Genetic Algorithms (GA), Particle Swarm Optimization (PSO), and Ant Colony Optimization (ACO) have been widely applied to address diverse routing scenarios, including static planning and dynamic adjustments.

An important approach for improving computational efficiency in large-scale problems is clustering, which involves grouping locations based on proximity or other criteria. K-means clustering is one of the most commonly used techniques for this purpose \cite{Alam_Sulaiman_2022}, dividing locations into well-defined regions to simplify route optimization. By reducing the problem size, clustering allows optimization algorithms like MILP, GA, or PSO to operate independently within each cluster, making the overall process more computationally feasible.

MILP excels at providing exact solutions by modeling objectives and constraints, such as visit frequencies and workload balancing, as linear functions. Although MILP is computationally expensive for large-scale problems, integrating it with K-means clustering significantly reduces complexity \cite{Laporte2017}.
DP, on the other hand, decomposes problems into smaller subproblems, solving each recursively and storing intermediate results. While useful for smaller PJP tasks, its computational demands make it more practical when combined with other methods for hybrid solutions \cite{feidiao_yang__2018}.

Metaheuristic algorithms like GA, PSO, and ACO provide robust alternatives for handling the multi-objective and dynamic nature of PJP optimization. GA iteratively refines solutions through evolutionary strategies like selection and crossover, effectively balancing travel distances and visit frequencies \cite{Konak_Coit_Smith_2006}. PSO, inspired by swarm behavior, optimizes routes by simulating collective decision-making and works particularly well when combined with clustering methods like K-means to refine regional solutions \cite{Hughes_Goerigk_Dokka_2020}. ACO, modeled on the foraging behavior of ants, is adept at exploring combinatorial solution spaces, avoiding local optima, and identifying efficient routes \cite{articleACO}.

Clustering and optimization algorithms work hand in hand for large-scale PJP problems. Clustering simplifies the problem by dividing it into smaller regions, while optimization techniques like GA, PSO, and MILP refine routes within each cluster. This integrated approach ensures computational efficiency, scalability, and practical applicability, making it well-suited for addressing PJP-specific challenges such as visit frequencies, even workload distribution, and time-sensitive constraints.

While researching for ways to solve the problem of creating optimal PJPs, we have come accross several approaches that solve a similar problem while optimizing other objectives that are discussed in detail below.

\subsection{Dynamic Programming (DP) to Solve VRPTW}
Dynamic Programming (DP) is a widely-used optimization technique capable of solving complex routing problems with constraints, making it particularly suitable for Permanent Journey Plan (PJP) optimization. Its ability to decompose a problem into smaller, manageable subproblems while ensuring global optimality makes it a valuable tool for addressing challenges such as visit frequencies, even visit spacing, time windows, and workload balancing.

A study \cite{articleDP} applied a DP algorithm to solve the Vehicle Routing Problem with Time Windows (VRPTW). This framework incrementally builds routes by sequentially adding stops while verifying feasibility under constraints. In a PJP context, this approach can be extended to align store visits with predefined schedules and ensure adherence to operating hours. The study reported a reduction in routes by 18\% and travel distances by over 5\%, highlighting the efficiency gains possible when dynamic adjustments are incorporated into route planning. For PJPs, such gains translate into increased store coverage and reduced operational costs.

Another study by Desaulniers et al. (2002) \cite{Desaulniers2002} focused on the use of DP to optimize fleet routing while accounting for time window constraints. Their approach incorporated a state-space relaxation method, reducing computational overhead while maintaining solution quality. For PJP, this method can ensure that sales representatives follow efficient schedules without exceeding working hour limits or violating store-specific time windows.

Furthermore, a study by Feillet et al. (2004) \cite{Feillet2004} introduced a label-setting algorithm based on DP to solve VRP with soft time windows. This approach allowed flexibility in time window adherence, a feature particularly useful for PJP scenarios where strict adherence to schedules may not always be possible. By penalizing minor deviations rather than rejecting infeasible solutions outright, this method ensures that PJPs remain robust under dynamic operational conditions.

A hybrid DP approach by Kilby et al. (2002) \cite{Kilby2002} combined DP with heuristic techniques to address large-scale routing problems. The hybrid method used DP to solve subproblems within clusters, which were formed using heuristic clustering methods. This is particularly relevant for PJPs, where clustering stores into smaller groups based on proximity or workload can significantly reduce the computational complexity of route planning.

% The study \cite{articleDP} develops a dynamic programming (DP) algorithm to solve the Vehicle Routing Problem with Time Windows (VRPTW) while adhering to European social legislation. It builds on the restricted DP framework where customers are sequentially added to partial vehicle routes, and feasibility is checked using additional state dimensions. To include compliance with drivers’ working and driving hour regulations, state dimensions are expanded with break scheduling.

% The proposed break scheduling algorithm operates locally (i.e., at or before adding a customer), scheduling breaks in waiting times due to hard time windows. Two methods are presented: a basic version focusing on local flexibility and an extended version incorporating optional rules from the legislation. This approach allows breaks and rests to be scheduled without increasing the overall time complexity of the DP algorithm.

% The proposed solution method outperforms state-of-the-art heuristics for the VRPTW with EC social legislation. The basic method reduced the average number of vehicle routes by over 18\% and the average travel distance by over 5\%, all while being computationally more efficient. The study highlights the significant impact of Directive 2002/15/EC on solutions and the importance of including optional rules, which contribute to further cost reductions (e.g., reducing vehicle count by 4\% and travel distance by 1.5\%). This emphasizes the practical value of applying these optional rules in route planning.

\subsection{Mixed-Integer Linear Programming (MILP) for PJP}
Mixed-Integer Linear Programming (MILP) is a mathematical optimization technique widely used in route planning and scheduling problems. It is particularly relevant for optimizing PJPs, where the focus is on visiting stores to generate demand while adhering to constraints such as visit frequencies, time windows, workload balancing, and minimizing travel distances.

MILP allows for the formulation of optimization problems as linear objective functions with linear constraints. For PJP scenarios, the objective might involve minimizing the total distance traveled or the time taken, while constraints ensure adherence to store-specific visit frequencies, consistent assignments for sales representatives, and balanced workloads. Unlike heuristic methods, MILP offers mathematically optimal solutions, making it a valuable tool for small to medium-scale PJP problems \cite{Laporte2017}.

A study by Sierksma and Tijssen (1998) highlights the use of MILP in sales territory alignment and routing problems, which are closely related to PJP optimization. Their approach ensures balanced workload distribution among sales representatives and minimizes travel costs \cite{Sierksma1998}.

For larger problems, MILP is often combined with clustering techniques to break down the problem into smaller subproblems. In such cases, clustering algorithms segment stores into manageable groups based on geographical proximity or other criteria, and MILP is applied within each cluster to generate optimal routes \cite{Nagy2007}. This hybrid approach improves computational feasibility while maintaining solution quality.

Despite its computational intensity, MILP serves as an excellent benchmark for evaluating heuristic or metaheuristic approaches, such as Genetic Algorithms (GA) and Particle Swarm Optimization (PSO). For example, a study by Schneider et al. (2014) demonstrated that MILP models could effectively optimize vehicle routing problems with time windows, which is directly applicable to PJP scenarios involving time-sensitive store visits \cite{Schneider2014}.

\subsection{Metaheuristic Approaches for VRPTW}
Metaheuristic algorithms like Genetic Algorithms (GA) and Ant Colony Optimization (ACO), widely applied to solve the Vehicle Routing Problem with Time Windows (VRPTW), are also effective for optimizing Permanent Journey Plans (PJP) due to their ability to handle multi-objective and constraint-driven problems.
A range of single-objective and multi-objective approaches has been explored to address the Vehicle Routing Problem with Time Windows (VRPTW). Brock University conducted a study \cite{ombuki2006multi} comparing single-objective and multi-objective methods, emphasizing a non-biased approach without weighted sum scoring to avoid compromising between vehicle usage and travel distance. The genetic algorithm (GA) used demonstrated optimal or near-optimal results for clustered data but was less consistent for uniformly distributed customer locations. The multi-objective approach faced challenges with Pareto dominance, leading to suboptimal compromises and limited generalizability to complex real-world scenarios.

A novel Multi-Objective Evolutionary Algorithm (MOEA) incorporating Jaccard’s coefficient was proposed to improve population diversity and Pareto-based approximations \cite{Garcia-Najera_Bullinaria_2011}. This method outperformed single-objective algorithms like EA with deterministic or randomized selection, achieving 2\% to 5\% better outcomes. However, its scalability was limited when applied to datasets with higher travel distances, posing challenges for large or complex datasets due to increased computation time.

A bi-objective GA combined with goal programming was designed to minimize vehicle use and travel costs, leveraging Pareto ranking and fitness evaluation with Pareto ranks \cite{Ghoseiri_Ghannadpour_2010}. This method showed positive results for clustered data, correlating vehicle count with travel cost. Yet, it struggled with unclustered datasets and mixed scenarios, which led to higher costs and limited effectiveness for larger, dynamically constrained problems.

The Hybrid Ant Colony Optimization (HACO) approach \cite{Zhang_Zhang_Ma_Zhang_Liu_2019} aimed to solve the multi-objective vehicle routing problem with flexible time windows (MOVRPFlexTW) by balancing customer satisfaction and cost. HACO used a hybrid strategy with multiple mutation operators and incorporated Pareto optimality, achieving good convergence speed and accuracy in Solomon’s benchmark problems. While HACO demonstrated robustness and adaptability to higher-dimensional data, finding optimal parameters was sensitive, potentially impacting efficiency and scalability for larger datasets.

These GA-based strategies each have strengths and limitations. The Brock University study highlighted trade-offs and biases when using Pareto dominance in multi-objective approaches. MOEA's incorporation of Jaccard’s coefficient improved diversity but faced scalability issues with higher travel distances. The bi-objective GA showed potential with clustered data but lacked stability in non-clustered scenarios. HACO, in contrast, outperformed traditional GAs and EAs, showing potential for real-world application and better scalability. Collectively, these findings highlight the need for further refinement and hybrid solutions to enhance generalizability and computational efficiency for real-world VRPTW problems.


\subsection{Hybrid Clustering Algorithms}

Another method to approach the VRP problem is to combine optimization techniques with clustering algorithms to solve complex VRP problems under real-life constraints. The clustering algorithms will be covered in the sections below.

Various clustering and optimization algorithms have been developed to address the complexities of Vehicle Routing Problems (VRPs). A three-phase algorithm approach from paper \cite{3_phase_cluster} that uses clustering, Mixed-Integer Linear Programming (MILP), and a Traveling Salesman Problem (TSP) approach has proven effective, reducing transport costs by 10\%-20\% in real-world cases. This method clusters customers into manageable subproblems, optimizes routes using MILP, and fine-tunes them with a TSP approach. However, its performance depends on initial clustering and struggles with large-scale problems.

Another approach \cite{firefly} uses the Discrete Firefly Algorithm (DFA) for clustering and a modified 2-opt algorithm for route optimization. DFA outperformed k-Means and manual methods, achieving up to a 15\% cost reduction and faster computation. Its main drawback is the dependence on initial clustering quality and computational intensity.

A hybrid algorithm \cite{hybrid_kmeans} for VRPs with Time Windows (VRPTW) begins with K-means clustering and refines it using OPTICS. Routes are generated using a Nearest Neighbor heuristic and optimized with 2-opt, showing a 5.06\% distance reduction for smaller instances and 11.97\% for larger ones. Its limitations include assumptions of static conditions and a need for parameter tuning.

The KACO algorithm for the Dynamic Location Routing Problem (DLRP) discussed in paper \cite{aco_dp_cluster} integrates K-means clustering with Ant Colony Optimization (ACO), using immigrant schemes to adapt routes dynamically. It outperformed traditional ACO, showing better tour lengths and adaptability under dynamic conditions. However, challenges with clustering quality and high computational demands remain.

These approaches demonstrate that combining clustering with optimization methods effectively handles VRPs, offering cost and efficiency benefits as compared to only adapting optimization algorithms to solve VRP problems. Moreover, they showcase better adaptability and scalability to real-world datasets as opposed to algorithms that solely relied on optimization to solve VRP that could not perform well with large and real-world datasets.


Drawing from the extensive range of methodologies explored, it is evident that optimization algorithms and clustering techniques play a crucial role in addressing complex routing and scheduling challenges across various domains. Genetic Algorithms (GA) have demonstrated their utility in multi-objective optimization, balancing competing factors such as distance minimization and timely service delivery through evolutionary processes. Similarly, Particle Swarm Optimization (PSO), inspired by the collective behavior of swarms, offers computational efficiency and adaptability in determining optimal routes and clustering solutions. Ant Colony Optimization (ACO), with its biologically inspired probabilistic approach, has proven adept at solving combinatorial problems like the Vehicle Routing Problem (VRP) by navigating through solution spaces to find optimal paths while evading local optima. Dynamic Programming (DP), although computationally intensive, provides exact solutions to problems by systematically breaking them into smaller subproblems.

The integration of these algorithms with clustering techniques, such as those used for grouping locations based on proximity and service requirements, enhances the efficiency of solving the vehicle routing and scheduling problems. Techniques like hierarchical clustering, k-means clustering, and density-based spatial clustering play pivotal roles in defining clusters that optimize routing paths, reduce travel times, and ensure that each route adheres to time windows and service constraints. These clustering approaches, combined with the optimization algorithms, help in forming practical, real-world solutions for complex logistical challenges. Through this literature review, it is clear that continued research and innovation in these areas are vital for developing more robust and scalable solutions to complex optimization problems, ultimately enhancing the efficiency and effectiveness of route optimization and planning in various industries.









