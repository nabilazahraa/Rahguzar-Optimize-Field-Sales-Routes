
The methodology for Rahguzar was designed as a structured, research-driven response to the operational inefficiencies in manual Permanent Journey Plan (PJP) scheduling. The project aimed to generate optimized PJPs that adhered to real-world constraints in the FMCG distribution landscape of Pakistan. Given the scale and complexity of the problem, we adopted a modular and algorithmically hybrid approach, structured into three main phases: Clustering, Scheduling, and Route Optimization.

Our methodology combined custom algorithm development with empirical evaluation on real-world datasets. This allowed us to balance theoretical rigor with practical usability, ensuring that the final solution could be deployed under real operational conditions.


\section{Problem Formulation}

The PJP problem is modeled as a weekly, constraint-aware variation of the VRPTW. Given a set of stores with known coordinates, visit frequencies, service durations, and territory assignments, the objective is to generate an optimized journey plan for each order booker such that:

\begin{itemize}
  \item Store visits satisfy required weekly frequencies
  \item Visits are spaced evenly across working days
  \item Total route time (travel + service) per day remains within shift limits (typically 8 hours)
  \item Order booker workload is balanced across the team
  \item Routes are geographically compact and operationally realistic
\end{itemize}

Due to the NP-hard nature of the problem, exact optimization methods are impractical at city scale. Instead, we use a hybrid, heuristic-based approach structured into three phases.

\section{Three-Phase Algorithmic Architecture}

\subsection{Clustering}

The clustering phase assigns stores to order bookers through a hybrid clustering pipeline that balances geographic proximity with workload equity. It includes:

\begin{itemize}
  \item \textbf{Graph-Based Clustering:} A minimum spanning tree (MST) is constructed from store locations to group nearby stores into initial clusters.
  \item \textbf{K-Means Refinement:} Each graph-based cluster is refined by minimizing intra-cluster variance. The centroid $C_i$ of each cluster is updated as:
  \[
  C_i = \frac{1}{N_i} \sum_{x \in S_i} x
  \]
  \item \textbf{Outlier Reassignment:} Stores that lie significantly far from their assigned cluster centroid are identified using:
  \[
  D_{s,c} > \mu + \sigma \cdot k
  \]
  and reassigned to the nearest neighboring cluster.
  \item \textbf{Workload Balancing:} Store workloads are redistributed by minimizing the squared workload error:
  \[
  SSE = \sum_{c} (W_c - \bar{W})^2
  \]
\end{itemize}

\subsection{Scheduling}

After clustering, each order booker's list of stores is scheduled across the working week using a custom \textit{Evolutionary Algorithm (EA)}. The EA process includes:

\begin{itemize}
  \item \textbf{Population Initialization:} Initial schedules are generated randomly and based on geographic heuristics.
  \item \textbf{Fitness Evaluation:} Each schedule is evaluated using:
  \[
  F = T_{\text{total}} + P_{\text{mismatch}} + P_{\text{imbalance}} + P_{\text{geo}}
  \]
  \item \textbf{Constraints:} Daily workload must satisfy:
  \[
  T_{\text{day}} = \sum_{i=1}^{n} T_{\text{travel},i} + \sum_{i=1}^{n} T_{\text{service},i} \leq 480
  \]
  \item \textbf{Crossover and Mutation:} Tournament selection, crossover, and mutation operations are used to refine schedules across generations.
\end{itemize}

\subsection{Route Optimization}

For each day’s assigned store list, visits are sequenced using \textit{Ant Colony Optimization (ACO)}. The probability of visiting store $j$ from $i$ is:

\[
P_{ij} = \frac{(\tau_{ij})^\alpha (\eta_{ij})^\beta}{\sum_{k \in N_i} (\tau_{ik})^\alpha (\eta_{ik})^\beta}
\]

Pheromone trails are updated as:
\[
\tau_{ij} = (1 - \rho) \cdot \tau_{ij} + \sum_{\text{ants}} \Delta \tau_{ij} \quad \text{where} \quad \Delta \tau_{ij} = \frac{Q}{L_k}
\]


\noindent Parameters used:
\begin{itemize}
  \item $\alpha = 1$, $\beta = 3$ (pheromone and distance influence)
  \item $\rho = 0.1$ (evaporation rate), $Q = 1$ (pheromone deposit factor)
  \item $L_k$: route length of ant $k$
\end{itemize}

Distances and travel times are computed using the OSRM backend for real-world accuracy.

\section{Data Preprocessing Strategy}

The input datasets were provided by SalesFlo in Excel format. The preprocessing pipeline involved:

\begin{itemize}
  \item Parsing and cleaning using Python
  \item Merging datasets by unique store identifiers
  \item Migrating structured data into a PostgreSQL database
  \item Filtering stores by operational zones for the pilot phase
\end{itemize}

This ensured consistency and readiness for integration into the optimization pipeline.

\section{Assumptions}

To maintain focus and computational tractability, the following assumptions were made:

\begin{itemize}
  \item Static average travel times are used; live traffic is not modeled
  \item Each route begins and ends at a fixed distributor
  \item Store service durations are known and fixed
  \item Planning is performed on a weekly horizon excluding Sundays
\end{itemize}

\section{Algorithmic Complexity}

Rahguzar’s modular pipeline breaks the PJP problem into three heuristically optimized stages: clustering, scheduling, and routing. While the overall problem is NP-hard, each component is designed to ensure scalability through local heuristics and bounded iterations.

\begin{itemize}
  \item \textbf{Clustering:} Graph-based clustering is performed using pairwise haversine distances with complexity $O(n^2)$, followed by K-Means refinement ($O(nki)$) and outlier reassignment ($O(n)$). This phase is executed once and scales well for realistic values of $n$ (number of stores) and $k$ (number of order bookers).

  \item \textbf{Scheduling (EA):} The Evolutionary Algorithm runs for $G_{EA}$ generations (typically 80–100) with a population size $P$ (typically 40–50). Fitness evaluation includes total route time, visit frequency matching, and workload balancing. Complexity per generation is $O(P \cdot n)$, giving:
  \[
  O(P \cdot n \cdot G_{EA})
  \]

  \item \textbf{Routing (ACO):} For each daily route of $r$ stores (where $r$ is typically 10–25), Ant Colony Optimization runs $m$ ants (usually 6–7) for $G_{ACO}$ generations (15–20 iterations). The per-route complexity is:
  \[
  O(m \cdot r^2 \cdot G_{ACO})
  \]
  Over all $n$ stores (divided across clusters and days), the total routing complexity is approximated as:
  \[
  O(n \cdot m \cdot r \cdot G_{ACO})
  \]
\end{itemize}

\textbf{Total System Complexity:}

Combining the three stages, the total time complexity of Rahguzar’s algorithm is:

\[
O(n^2 + nk + P \cdot n \cdot G_{EA} + n \cdot m \cdot r \cdot G_{ACO})
\]

This formulation reflects both theoretical bounds and empirical performance. In practice, early stopping, parallelism, and per-cluster decomposition significantly reduce runtime. Rahguzar consistently generated optimized plans for hundreds of stores in under 5 minutes on a standard multi-core machine.
