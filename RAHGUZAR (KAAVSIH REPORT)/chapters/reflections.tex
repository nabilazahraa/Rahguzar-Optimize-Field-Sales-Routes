The Rahguzar project presented an opportunity not only to apply theoretical concepts in a practical setting but also to experience the real-world challenges of system development, collaboration, and research-driven problem solving. This section provides a comprehensive reflection on individual contributions, team dynamics, and process evolution throughout the project lifecycle. By analyzing the gap between initial plans and actual achievements, we aim to capture the key lessons learned and areas for personal and collective growth.

\section{Individual Reflections}

\begin{itemize}
    \item \textbf{Iqra Azfar}: Over the duration of the project, I was not only dedicated to doing extensive research and examining different facets of the issue, but also to developing essential technical and soft skills on an ongoing basis. The methodology included picking up new tools, studying intricate concepts, and implementing them to solve real-world problems. Collaboration with my team members was an enriching experience, where I actively participated in brainstorming meetings, solution creation, and decision-making. Working with others showed me the value of effective communication, flexibility, and respect towards one another, particularly in dealing with conflicts or differing opinions. These experiences helped me learn more about teamwork, problem-solving, and lifelong learning.
    \item \textbf{Muhammad Youshay}:
    \item \textbf{Rabia Shahab}:
    \item \textbf{Nabila Zahra}
\end{itemize}

\section{Team Reflection}
% Collaboration experience, role distribution, communication, and teamwork challenges.
Working together on Rahguzar was a deeply collaborative experience that taught us the value of shared vision, consistent communication, and mutual accountability.
While we were united by a shared goal, the process of getting there involved navigating a variety of challenges that tested our coordination, communication, and adaptability. One of the most significant challenges we encountered was working at different paces. Each team member had their own academic and personal commitments, and synchronizing our work schedules proved difficult at times. There were also disagreements on how to approach certain parts of the problem, particularly in algorithm design, interface behavior, and system architecture.
At one point, we struggled to align our approaches to problem-solving, particularly in how to structure the optimization logic, divide backend responsibilities, and sequence integration with the frontend. Each of us had different interpretations of the problem and varying ideas on how best to implement certain features. These disagreements sometimes led to delays and overlapping efforts. However, they also turned out to be important learning opportunities. What helped us move forward was our ability to step back, have honest discussions, and actively listen to one another. The guidance and support we received from our supervisor and mentors during these phases were especially valuable. Their input helped us untangle complex decisions, identify a shared direction, and turn moments of friction into productive design conversations.
Despite the differences in working styles and technical opinions, we learned to trust each other’s strengths and make space for every perspective. As the project progressed, we grew more coordinated and developed a rhythm that allowed us to work more effectively together. Ultimately,Rahguzar was not just a technical achievement, but a product of collective resilience, trust, and shared commitment to solving a real-world problem together.
\section{Process Reflection}
% What worked well? What could be improved in the your project?
Looking back at the process, one of the things that worked particularly well was our decision to follow an iterative and research-driven development approach. Early in the project, we focused on understanding the problem deeply through industry interviews, literature review, and small-scale experiments. This helped us avoid jumping prematurely into implementation and instead guided our design with clarity and purpose.
Another strength was our consistent engagement with real-world data and constraints. By validating our design choices against actual operational needs provided by SalesFlo, we ensured that our solution stayed grounded and practical.

However, we also faced challenges that impacted our efficiency. A major bottleneck occurred when all of us were working on different parts of the codebase simultaneously. While this parallel effort helped accelerate feature development, it made integration complex and time-consuming. Merging work from multiple branches while ensuring compatibility and system stability required multiple debugging sessions and technical compromises. In hindsight, a more modular and version-controlled integration plan from the start could have reduced friction.

We also underestimated the time needed for certain tasks, particularly data integration and testing. Delays in data availability affected our ability to validate early outputs, and unexpected inconsistencies in the dataset required additional preprocessing.

Despite these challenges, the process evolved and improved over time. We became more deliberate in dividing tasks, more consistent in communication, and more disciplined in tracking progress. This experience highlighted the importance of not just technical knowledge, but also process design and team coordination in building real-world systems.

\section{Plan vs Achievment}
% What was proposed vs what was achieved? 
% Were any proposed features removed/were any features added that were not proposed initially? 
% What are the reasons for these choices? 
% What are opportunities and challenges led to these design decisions? 
In the initial stages of development and testing, the project encountered several challenges, particularly in the selection and evaluation of appropriate algorithms. One of the key difficulties arose from operational constraints provided by SalesFlo, including specific scheduling rules based on store types. For example, wholesale stores required a minimum gap of two days between consecutive visits. These requirements introduced additional complexity into the scheduling logic that was not fully anticipated during the planning phase.

The original design included the use of the Google Maps API for generating distance matrices essential for route optimization. However, during implementation, we faced quota limitations and cost-related constraints that made it unsustainable for repeated large-scale use. As a result, we migrated to the Open Source Routing Machine (OSRM), which offered a cost-effective and reliable alternative. OSRM integrated smoothly with the backend system and allowed us to maintain routing functionality without dependency on third-party paid services.

Additionally, the performance metrics defined at the start of the project were revised. Initially, we focused on general algorithm efficiency. As the project progressed, these metrics were refined to better reflect the quality of the generated Permanent Journey Plans (PJPs). New evaluation criteria included factors such as adherence to visit frequency requirements, practical feasibility of routes, and overall balance in daily assignments. These changes were informed by observations during pilot testing and helped ensure that system outputs aligned more closely with operational expectations.

Another major change involved the dashboard technology. The initial plan was to use Power BI for data visualization. However, Power BI presented limitations in terms of scalability, customization, and backend integration. To address this, we transitioned to a custom-built solution using React for the frontend and Flask for the backend. This provided improved control over interface design and enabled real-time communication with the core system. During this process, an additional feature was also introduced: the ability to generate journey plans for extended planning windows while still meeting the original constraints—such as visit frequency, rest-day intervals, and fair workload distribution across order bookers.

These design changes reflect an adaptive and iterative approach guided by technical constraints and practical deployment considerations. While some originally proposed features were modified or replaced, new capabilities were added to enhance system performance and usability. Overall, the project evolved in a way that better aligned with both user requirements and system scalability, resulting in a more robust and deployment-ready solution