The Rahguzar project presented an opportunity not only to apply theoretical concepts in a practical setting but also to experience the real-world challenges of system development, collaboration, and research-driven problem solving. This section provides a comprehensive reflection on individual contributions, team dynamics, and process evolution throughout the project lifecycle. By analyzing the gap between initial plans and actual achievements, we aim to capture the key lessons learned and areas for personal and collective growth.

\section{Individual Reflections}

\begin{itemize}
    \item \textbf{Nabila Zahra}: Working on Rahguzar has been one of the most rewarding experiences of my academic life, helping me grow both technically and personally. It was the first time I contributed to a project of this scale, developed in close collaboration with an industry partner—SalesFlo—where every component, from algorithm design to deployment, had real-world implications. This made the journey both challenging and rewarding.

    One of the most meaningful parts of the project for me was working on the scheduling system using an Evolutionary Algorithm. I had previously studied EAs in my Computational Intelligence course at Habib University, but this project allowed me to apply that theoretical knowledge in a much more complex and realistic setting. Translating business constraints into fitness functions, tuning parameters based on performance, and integrating the scheduler into a broader optimization pipeline helped me understand the practical challenges of implementing heuristic algorithms. It reinforced that academic understanding is only the starting point—real-world systems require adaptability, debugging, and iteration beyond what textbooks teach.
    
    Beyond the algorithm itself, I worked extensively on both the backend and frontend of our application. I developed APIs to serve the optimized schedules and routing data, and helped integrate them with our interactive map interface. Seeing the UI reflect our algorithmic output in real time was incredibly satisfying—it reminded me why I enjoy building end-to-end systems.
    
    I also stepped out of my comfort zone by handling deployment tasks. I learned how to use Docker to containerize OSRM, and how to set up and configure AWS EC2 and RDS with PostgreSQL. These were tools I had not worked with before, and figuring them out taught me how to be resourceful and persistent when dealing with unfamiliar systems. At times, debugging deployment issues felt frustrating, but I now see how those moments pushed me to grow the most.
    
    What stood out to me throughout this project was how much value lies in cross-functional thinking—understanding how the backend affects the frontend, how the database interacts with APIs, and how deployment choices affect performance. Working on Rahguzar helped me connect these dots and become more confident in navigating complex, integrated systems.
    
    Most importantly, this project reminded me that challenges are learning curves. Whether it was refining the EA scheduler, troubleshooting server errors, or adapting designs based on feedback, each obstacle became an opportunity to learn, grow, and build something better. I walk away from this project with a deeper appreciation for teamwork, a stronger technical foundation, and a clearer vision of the kind of systems I want to build in the future.

    \item \textbf{Iqra Azfar}: Over the duration of the project, I was not only dedicated to doing extensive research and examining different facets of the issue, but also to developing essential technical and soft skills on an ongoing basis. The methodology included picking up new tools, studying intricate concepts, and implementing them to solve real-world problems. Collaboration with my team members was an enriching experience, where I actively participated in brainstorming meetings, solution creation, and decision-making. Working with others showed me the value of effective communication, flexibility, and respect towards one another, particularly in dealing with conflicts or differing opinions. These experiences helped me learn more about teamwork, problem-solving, and lifelong learning.
    \item \textbf{Muhammad Youshay}: Embarking on this project was an enriching experience, particularly due to the opportunity to solve an actual industry problem presented by SalesFlo in the FMCG sector—a field entirely new to me. This required a deep understanding of the intricacies involved in permanent journey plan (PJP) scheduling, pushing me to thoroughly analyze operational constraints, optimization criteria, and stakeholder needs. The Human-Centered Design (HCD) approach was central to our methodology, enabling us to systematically empathize with end-users, define core problems, ideate viable solutions, prototype concepts, and iterate based on real-world feedback.

    The experience significantly expanded my skill set. Leveraging my academic knowledge from Habib University's courses, particularly Computational Intelligence, was crucial in developing the evolutionary algorithm component of our project. This course provided the theoretical foundation and practical techniques that enabled the design and fine-tuning of our hybrid optimization approach, effectively addressing complex, multi-objective problems. Additionally, my problem-solving abilities and logic-building skills greatly improved as I navigated challenges such as workload balancing, geographic clustering, and routing optimization. Exploring the problem from different perspectives and applying diverse analytical lenses was particularly enlightening.
    
    From a technical standpoint, I enhanced my proficiency in Python, OR-Tools, and data preprocessing techniques. Furthermore, interacting extensively with PostgreSQL and cloud-based systems like AWS strengthened my database management and backend development capabilities, preparing me for more complex technical tasks in the future.
    
    Working in a team presented both enriching experiences and notable challenges. Our team brought together individuals with different ideas, approaches, and technical expertise. Achieving consensus required effective communication, open-mindedness, and patience. Navigating through varied perspectives and integrating everyone's inputs to form cohesive solutions taught me invaluable lessons in teamwork, collaboration, and leadership.
    
    Overall, the experience was immensely rewarding, significantly boosting both my technical expertise and soft skills. I have emerged more confident in my ability to approach complex real-world problems with structured, innovative solutions, and equipped to thrive in collaborative environments.
    \item \textbf{Rabia Shahab}: Working on Rahguzar has been one of the most impactful experiences of my undergraduate journey, offering a chance to apply technical skills to a meaningful real-world problem. Collaborating with SalesFlo introduced a level of complexity and accountability that shaped how we approached both design and implementation. Our work wasn't just theoretical—it had to make sense in a business context and stand up to practical constraints, which made the learning curve steep but highly rewarding.

    My core contributions focused on the areas of developing the Evolutionary Algorithm (EA) scheduler and designing a performance dashboard. The EA scheduler pushed me to move beyond textbook models and think critically about how to encode real-world business requirements—such as fairness in visit distribution and route efficiency—into algorithmic logic. Balancing multiple objectives, tuning parameters, and debugging unexpected behaviors taught me that optimization in real systems is messy, and that iteration is essential.
    
    Simultaneously, building the dashboard gave me a deeper appreciation for data interpretation and user-centered design. It helped me identify and prioritize the KPIs that actually matter—things like distance coverage, visit gaps, and time allocation per resource. Creating a clear interface to display these metrics not only made the system more usable, but also helped bridge the communication gap between the technical team and non-technical stakeholders and enable them to be more data-driven. It was a reminder that even the best algorithm needs to be transparent and interpretable to deliver real value.
    
    Throughout the project, I also explored new technical tools that were initially unfamiliar. I worked with OSRM to integrate routing functionality and used Docker to containerize services for a smoother deployment process. These experiences gave me hands-on exposure to backend infrastructure and helped me understand how different system components—schedulers, route engines, data pipelines—must work together in sync.
    
    Equally important was the experience of working with my teammates. We each brought different strengths to the table, and there were moments where progress felt uncertain—whether due to technical issues or conflicting ideas. However, through open discussion, shared debugging sessions, and brainstorming under pressure, we consistently found ways forward. I learned that collaboration isn't just about dividing tasks—it's about supporting each other, challenging assumptions, and finding common ground when faced with complexity.
    
    Rahguzar taught me that building real systems requires not just technical skills but also adaptability, empathy, and trust in your team. I leave this project with greater confidence in my ability to contribute to large, interdisciplinary efforts, and a stronger sense of the kind of systems and teams I want to be a part of in the future.
    
\end{itemize}

\section{Team Reflection}
% Collaboration experience, role distribution, communication, and teamwork challenges.
Working together on Rahguzar was a deeply collaborative experience that taught us the value of shared vision, consistent communication, and mutual accountability.
While we were united by a shared goal, the process of getting there involved navigating a variety of challenges that tested our coordination, communication, and adaptability. One of the most significant challenges we encountered was working at different paces. Each team member had their own academic and personal commitments, and synchronizing our work schedules proved difficult at times. There were also disagreements on how to approach certain parts of the problem, particularly in algorithm design, interface behavior, and system architecture.
At one point, we struggled to align our approaches to problem-solving, particularly in how to structure the optimization logic, divide backend responsibilities, and sequence integration with the frontend. Each of us had different interpretations of the problem and varying ideas on how best to implement certain features. These disagreements sometimes led to delays and overlapping efforts. However, they also turned out to be important learning opportunities. What helped us move forward was our ability to step back, have honest discussions, and actively listen to one another. The guidance and support we received from our supervisor and mentors during these phases were especially valuable. Their input helped us untangle complex decisions, identify a shared direction, and turn moments of friction into productive design conversations.
Despite the differences in working styles and technical opinions, we learned to trust each other’s strengths and make space for every perspective. As the project progressed, we grew more coordinated and developed a rhythm that allowed us to work more effectively together. Ultimately,Rahguzar was not just a technical achievement, but a product of collective resilience, trust, and shared commitment to solving a real-world problem together.
\section{Process Reflection}
% What worked well? What could be improved in the your project?
Looking back at the process, one of the things that worked particularly well was our decision to follow an iterative and research-driven development approach. Early in the project, we focused on understanding the problem deeply through industry interviews, literature review, and small-scale experiments. This helped us avoid jumping prematurely into implementation and instead guided our design with clarity and purpose.
Another strength was our consistent engagement with real-world data and constraints. By validating our design choices against actual operational needs provided by SalesFlo, we ensured that our solution stayed grounded and practical.

However, we also faced challenges that impacted our efficiency. A major bottleneck occurred when all of us were working on different parts of the codebase simultaneously. While this parallel effort helped accelerate feature development, it made integration complex and time-consuming. Merging work from multiple branches while ensuring compatibility and system stability required multiple debugging sessions and technical compromises. In hindsight, a more modular and version-controlled integration plan from the start could have reduced friction.

We also underestimated the time needed for certain tasks, particularly data integration and testing. Delays in data availability affected our ability to validate early outputs, and unexpected inconsistencies in the dataset required additional preprocessing.

Despite these challenges, the process evolved and improved over time. We became more deliberate in dividing tasks, more consistent in communication, and more disciplined in tracking progress. This experience highlighted the importance of not just technical knowledge, but also process design and team coordination in building real-world systems.

% \section{Plan vs Achievment}
% What was proposed vs what was achieved? 
% Were any proposed features removed/were any features added that were not proposed initially? 
% What are the reasons for these choices? 
% What are opportunities and challenges led to these design decisions? 
% In the initial stages of development and testing, the project encountered several challenges, particularly in the selection and evaluation of appropriate algorithms. One of the key difficulties arose from operational constraints provided by SalesFlo, including specific scheduling rules based on store types. For example, wholesale stores required a minimum gap of two days between consecutive visits. These requirements introduced additional complexity into the scheduling logic that was not fully anticipated during the planning phase.

% The original design included the use of the Google Maps API for generating distance matrices essential for route optimization. However, during implementation, we faced quota limitations and cost-related constraints that made it unsustainable for repeated large-scale use. As a result, we migrated to the Open Source Routing Machine (OSRM), which offered a cost-effective and reliable alternative. OSRM integrated smoothly with the backend system and allowed us to maintain routing functionality without dependency on third-party paid services.

% Additionally, the performance metrics defined at the start of the project were revised. Initially, we focused on general algorithm efficiency. As the project progressed, these metrics were refined to better reflect the quality of the generated Permanent Journey Plans (PJPs). New evaluation criteria included factors such as adherence to visit frequency requirements, practical feasibility of routes, and overall balance in daily assignments. These changes were informed by observations during pilot testing and helped ensure that system outputs aligned more closely with operational expectations.

% Another major change involved the dashboard technology. The initial plan was to use Power BI for data visualization. However, Power BI presented limitations in terms of scalability, customization, and backend integration. To address this, we transitioned to a custom-built solution using React for the frontend and Flask for the backend. This provided improved control over interface design and enabled real-time communication with the core system. During this process, an additional feature was also introduced: the ability to generate journey plans for extended planning windows while still meeting the original constraints—such as visit frequency, rest-day intervals, and fair workload distribution across order bookers.

% These design changes reflect an adaptive and iterative approach guided by technical constraints and practical deployment considerations. While some originally proposed features were modified or replaced, new capabilities were added to enhance system performance and usability. Overall, the project evolved in a way that better aligned with both user requirements and system scalability, resulting in a more robust and deployment-ready solution

\section{Plan vs Achievement}


The initial vision for the Rahguzar system was shaped by academic research, stakeholder discussions, and a focus on building a modular, data-driven solution for journey plan optimization. While the core objectives remained consistent throughout the project, many aspects of the system evolved significantly during development. These changes were driven by technical constraints, real-world deployment challenges, and feedback from pilot testing. As a result, several design decisions were revised, new features were added, and some proposed components were removed—ultimately leading to a more practical, efficient, and scalable system.
\\
\textbf{Algorithm Design:} Initially, a basic heuristic-based approach was proposed for assigning stores to days and order bookers. However, this proved insufficient to meet complex requirements such as visit frequency enforcement, workload balancing, and multi-day planning. We pivoted to a custom-built Evolutionary Algorithm (EA), which allowed encoding multiple operational constraints into a single fitness function. This shift enabled more flexible and scalable scheduling under real-world conditions.
\\
\textbf{Routing Engine:} The original system design relied on the Google Maps API for generating travel distances and times. During development, cost limitations and API quotas led us to migrate to a Dockerized, self-hosted instance of the Open Source Routing Machine (OSRM), which offered a scalable, open-source alternative. OSRM integrated seamlessly into our Flask backend and provided accurate routing using OpenStreetMap data, supporting high-performance route sequencing at no additional cost.
\\
\textbf{Technology Stack:} While Power BI was initially considered for dashboard visualization, it lacked the flexibility needed for real-time updates and seamless backend integration. This limitation prompted a shift to a custom React-based dashboard, with Flask serving data endpoints. The result was a more interactive and responsive UI, with tighter control over both the visual and functional layers of the system.
\\
\textbf{Data Management:} Contrary to initial assumptions, store data and geographic metadata required substantial preprocessing. Duplicates, missing coordinates, and inconsistent fields had to be addressed through custom scripts and validation steps. These preprocessing efforts became crucial to maintaining accuracy in clustering, scheduling, and route generation.
\\
\textbf{Integration Strategy:} Our original plan was to integrate all components toward the final phase of development. This approach proved problematic, especially as different modules evolved in parallel. As a result, we adopted a continuous integration model midway through the project, merging backend and frontend functionality incrementally to detect compatibility issues earlier.
\\
\textbf{Performance Metrics:} Initially, success was to be measured in terms of algorithm speed and distance minimization. However, real-world feedback highlighted the need for more meaningful metrics. We revised our KPIs to include visit frequency adherence, shift time compliance, workload standard deviation, and travel vs in-store time ratios. These metrics offered more actionable insights and improved the system’s operational relevance.
\\
\textbf{Feature Additions:}
\begin{itemize}
    \item \textbf{Custom Day Planning:} Instead of fixed weekly or monthly planning cycles, the system now allows users to define custom durations for journey plans, providing greater flexibility.
    \item \textbf{Automatic Extra Day Allocation:} If the number of available days is insufficient to accommodate all required store visits, the system intelligently adds extra working days to complete the plan while maintaining constraint compliance.
    \item \textbf{Dynamic Rerouting and Manual Overrides:} Managers can trigger rerouting in real time by modifying store assignments, improving adaptability.
\end{itemize}

\textbf{Feature Removals:}
\begin{itemize}
    \item \textbf{Sales Data Integration:} Early plans included integration with historical sales data to prioritize high-performing stores. This feature was removed due to unavailability of sales records and limited scope in the pilot phase.
    \item \textbf{Mobile Application Support:} Native mobile deployment was removed from the scope in favor of a responsive web-based interface, reducing complexity while retaining cross-device accessibility.
    \item \textbf{OB Recommendation System:} A proposed OB-store matching suggestion engine was deprioritized due to time constraints and shifting focus toward core scheduling logic.
\end{itemize}

In summary, while several originally proposed features were re-evaluated or removed, new capabilities were introduced that better addressed stakeholder needs and system scalability. These adaptive decisions reflect the project’s iterative development philosophy and contributed to a more robust, usable, and deployment-ready solution.
