\section{Individual Reflections}
% Personal learning experience, skills gained, and challenges overcome.

\section{Team Reflections}
% Collaboration experience, role distribution, communication, and teamwork challenges.
Working together on Rahguzar was a deeply collaborative experience that taught us the value of shared vision, consistent communication, and mutual accountability.
While we were united by a shared goal, the process of getting there involved navigating a variety of challenges that tested our coordination, communication, and adaptability. One of the most significant challenges we encountered was working at different paces. Each team member had their own academic and personal commitments, and synchronizing our work schedules proved difficult at times. There were also disagreements on how to approach certain parts of the problem, particularly in algorithm design, interface behavior, and system architecture.
At one point, we struggled to align our approaches to problem-solving, particularly in how to structure the optimization logic, divide backend responsibilities, and sequence integration with the frontend. Each of us had different interpretations of the problem and varying ideas on how best to implement certain features. These disagreements sometimes led to delays and overlapping efforts. However, they also turned out to be important learning opportunities. What helped us move forward was our ability to step back, have honest discussions, and actively listen to one another. The guidance and support we received from our supervisor and mentors during these phases were especially valuable. Their input helped us untangle complex decisions, identify a shared direction, and turn moments of friction into productive design conversations.
Despite the differences in working styles and technical opinions, we learned to trust each other’s strengths and make space for every perspective. As the project progressed, we grew more coordinated and developed a rhythm that allowed us to work more effectively together. Ultimately,Rahguzar was not just a technical achievement, but a product of collective resilience, trust, and shared commitment to solving a real-world problem together.
\section{Process Reflection}
% What worked well? What could be improved in the your project?
Looking back at the process, one of the things that worked particularly well was our decision to follow an iterative and research-driven development approach. Early in the project, we focused on understanding the problem deeply through industry interviews, literature review, and small-scale experiments. This helped us avoid jumping prematurely into implementation and instead guided our design with clarity and purpose.
Another strength was our consistent engagement with real-world data and constraints. By validating our design choices against actual operational needs provided by SalesFlo, we ensured that our solution stayed grounded and practical.

However, we also faced challenges that impacted our efficiency. A major bottleneck occurred when all of us were working on different parts of the codebase simultaneously. While this parallel effort helped accelerate feature development, it made integration complex and time-consuming. Merging work from multiple branches while ensuring compatibility and system stability required multiple debugging sessions and technical compromises. In hindsight, a more modular and version-controlled integration plan from the start could have reduced friction.

We also underestimated the time needed for certain tasks, particularly data integration and testing. Delays in data availability affected our ability to validate early outputs, and unexpected inconsistencies in the dataset required additional preprocessing.

Despite these challenges, the process evolved and improved over time. We became more deliberate in dividing tasks, more consistent in communication, and more disciplined in tracking progress. This experience highlighted the importance of not just technical knowledge, but also process design and team coordination in building real-world systems.

\section{Plan vs Achievment}
% What was proposed vs what was achieved? 
% Were any proposed features removed/were any features added that were not proposed initially? 
% What are the reasons for these choices? 
% What are opportunities and challenges led to these design decisions? 
