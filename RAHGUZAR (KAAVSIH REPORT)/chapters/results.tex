\section{Algorithmic Phases and Experiment Results}

This section presents the methodology and results from a series of experiments conducted to determine the optimal combinations of algorithms for different phases of the optimization process.




\subsection{Phase 1: Hybrid Clustering Approach}
% Phase 1 content to be added later.
The first aim of the algorithm is to divide the list of stores to be serviced into a number of clusters which is determined by the number of orderbookers.
The objective is to group close stores in the same cluster, so that the orderbookers can service them as the scheduler will decide.
To achieve this, clustering algorithms were tested on their own adn in a hybrid approach.

\subsubsection{Clustering Algorithms Tested}

The following clustering algorithms were evaluated for grouping stores among orderbookers based on geographical proximity and workload balancing:

\begin{itemize}
    \item \textbf{KMeans Clustering}
    \begin{itemize}
        \item Groups stores based on latitude and longitude using KMeans.
        \item Minimizes the sum of squared distances to cluster centroids.
        \item Post-clustering, stores are reallocated from overloaded to underloaded clusters to balance workload.
    \end{itemize}
    
    \item \textbf{Hierarchical Clustering}
    \begin{itemize}
        \item Uses Ward's linkage to merge clusters based on minimum variance increase.
        \item Clusters formed by cutting the dendrogram at desired number of orderbookers.
        \item Produces compact and cohesive geographic clusters.
    \end{itemize}
    
    \item \textbf{Gaussian Mixture Models}
    \begin{itemize}
        \item Models clusters as Gaussian distributions with flexible shapes and orientations.
        \item Assigns stores to clusters based on probability of belonging to a Gaussian component.
        \item Covariance type set to "full" to allow non-spherical clusters.
    \end{itemize}

    \item \textbf{Hybrid Approach 1 (Graph-Based Clustering + KMeans)}
    \begin{itemize}
        \item First forms clusters using graph-based connectivity within a distance threshold.
        \item Then refines cluster centers using KMeans to match desired number of clusters.
        \item Ensures geographic cohesion and adjusts for cluster count.
    \end{itemize}

    \item \textbf{Hybrid Approach 2 (Graph-Based Clustering + KMeans + Geographical Constraints)}
    \begin{itemize}
        \item Builds on Hybrid 1 by detecting and reassigning outlier stores.
        \item Outliers identified by distance from cluster centroid beyond a defined threshold.
        \item Reassigned to nearest cluster to improve cohesion and balance.
    \end{itemize}
\end{itemize}

Moreover, to ensure that optimal clusters were being made, cluster balancing was employed to redistribute 
stores between clusters to ensure workloads—based on service effort and travel time—are roughly equal.
 Overloaded clusters transfer nearby stores to underloaded ones, and this process repeats until workloads fall within a set tolerance or a maximum number of iterations is reached, promoting fair and efficient distribution.

\subsection{Phase 1: Experiments Results}
The silhouette score measures how well points are grouped in a clustering task. It checks if points are close to others in their own cluster, indicating a good score, or if they are far from points in other clusters (also good score). The score ranges from -1 to 1, where 1 means perfect clusters, 0 means overlap, and negative scores indicate that points might be in the wrong cluster.

\begin{table}[H]
    \centering
    \resizebox{\textwidth}{!}{%
    \begin{tabular}{|c|c|c|c|c|c|}
    \hline
    \textbf{Distributor ID} & \textbf{KMeans} & \textbf{Gaussian Mixture Models} & \textbf{Hierarchical} & \textbf{Hybrid Approach-1} & \textbf{Hybrid Approach-2} \\
    \hline
    1 & 0.3537 & 0.2835 & 0.3537 & 0.5044 & 0.5044 \\
    6 & 0.1839 & 0.2130 & 0.1839 & 0.1839 & 0.1684 \\
    7 & 0.2737 & 0.3376 & 0.3892 & 0.3980 & 0.4437 \\
    \hline
    \end{tabular}%
    }
    \caption{Silhouette Score Comparison for Different Clustering Algorithms}
    \label{tab:silhouette_scores}
    \end{table}

From the results in Table~\ref{tab:silhouette_scores}, it can be observed that the Hybrid Approach-2 outperformed all other clustering algorithms, achieving the highest silhouette score of 0.5044 for Distributor 1 and 0.4437 for Distributor 7. This indicates that the clusters formed using this approach were more cohesive and well-separated compared to those formed by the other algorithms.
Moreover, it finds a good balance between workload distribution and geographic proximity. For example, for distributor ID 7, there's a little variance in the clusters being formed however, it has the highest silhouette score as geographical proximity has been set as the priority.



\subsection{Phase 2: Scheduling with Evolutionary Algorithm (EA)}
Once clusters have been formed in Phase 1, the next critical step is to generate a feasible schedule that defines which shops should be visited on each day of the planning period (either a single day or a custom range, as selected by the user). The primary objective during scheduling is to balance the workload across both the available days and all assigned orderbookers. This ensures that no orderbooker is disproportionately burdened and that their daily tasks are distributed as evenly as possible.

To identify the most effective scheduling strategy, a series of experiments were conducted using various algorithms. Each algorithm was evaluated based on its ability to generate balanced, constraint-compliant schedules. At the core of this evaluation process is the fitness function.

The fitness function plays a vital role in determining how practical and efficient a schedule is. It computes a cost value for each candidate schedule—lower values indicate better solutions. Hard constraints are applied first: for example, if the total route time on any day exceeds a predefined daily limit (e.g., 480 minutes), a substantial penalty is applied to immediately discourage infeasible schedules. It also checks for visit mismatches—stores not visited the required number of times—penalizing any deviations.

Additionally, soft constraints are incorporated to further refine the schedule. These include a day-balancing penalty when daily route times fall outside the ideal range (e.g., 360–420 minutes), and a geographical penalty when neighboring stores requiring only one visit are assigned to different days. These constraints collectively guide the algorithm toward generating schedules that are not only valid but also optimized for balance, practicality, and real-world efficiency. The following scheduling algorithms were used:

\begin{itemize}
    \item Simulated Annealing
    \item Ant Colony Optimization (ACO)
    \item Mixed Integer Linear Programming (MILP)
    \item Evolutionary Algorithm (EA)
\end{itemize}

Through a series of experiments, the best combination for the scheduling phase was determined, which was Evolutionary Algorithm (EA) for scheduling.
Algorithms like ACO showed that the in serach for the optimal schedule, it can often get trapped in a local optima. For MILP and simulated annealing, 
generating a schedule can be computationally expensive and time consuming. On the other hand, EA showed optmial results with genetic diversity maintained that
avoids premature convergence, hence EA is the best choice for scheduling.

\subsection{Phase 3: Route Optimization with Ant Colony Optimization (ACO)}
After a schedule has been created, the next step is to optimize the routes for each orderbooker. This involves determining the most efficient path for each orderbooker to follow, ensuring that they can visit all assigned stores in the shortest possible time while adhering to any constraints (e.g., time windows, vehicle capacity). The goal is to minimize travel distance and time while maximizing efficiency.
This phase is crucial for ensuring that the orderbookers can complete their routes within the allocated time and resources, ultimately leading to improved service levels and reduced operational costs.
Moreover, it helps determine the order of visiting the stores and the best route to take, considering factors such as road networks and other logistical constraints.
For route optimization, several algorithms were tested to identify the most effective one. The route optimization algorithms evaluated were:

\begin{itemize}
    \item Particle Swarm Optimization (PSO)
    \item Ant Colony Optimization (ACO)
    \item Google Optimization Tools (OR-Tools)
    \item Evolutionary Algorithm (EA)
    \item Dynamic Programming
\end{itemize}

It was found that the Ant Colony Optimization (ACO) route optimizer performed the best across all tested route optimization algorithms.
While POS lead to explorative results, it would often get trapped in local optimas and not perform up to the optimal mark. Using Google OR-Tools and 
Dynamic Programming showed that performance would degrade due to the very large datasets and complex constraints involved. Whereas ACO's 
pheromone mechanisms lef to effective exploration of the solution space, allowing high-quality route optimization solutions in a reasonable time frame.


\subsection{Phase 2 and 3: Experiments Overview}
A series of experiments were conducted across different distributors with varying numbers of stores and orderbookers. The following distributors were involved in the experiments:
\begin{table}[h!]
\centering
\begin{tabular}{|c|c|c|}
\hline
\textbf{Distributor ID} & \textbf{Number of Stores} & \textbf{Number of Orderbookers} \\
\hline
1 & 395 & 2 \\
6 & 426 & 4 \\
7 & 580 & 5 \\
131 & 42 & 1 \\
\hline
\end{tabular}
\caption{Distributors and their Assigned Stores and Orderbookers}
\end{table}

The purpose of the experiments was to determine the best combinations of scheduling and route optimization algorithms for each distributor. The experiments were designed to evaluate the performance of different algorithmic combinations in terms of distance and travel time minimization.
This was done by analyzing the total distance and the total traveltime minimized across all orderbookers (results show average of all distributors), and the stores serviced distribution per orderbooker for each day.

\begin{figure}[H]
    \centering
    \includegraphics[width=0.95\textwidth]{images/results_distance_all_dis}
    \caption{Total distance for all distributors compared over different combinations.}
    \label{fig:results_distance_all_dis}
\end{figure}

\begin{figure}[H]
    \centering
    \includegraphics[width=0.95\textwidth]{images/results_time_all_dis}
    \caption{Total time for all distributors compared over different combinations.}
    \label{fig:results_time_all_dis}
\end{figure}

The results in Figure~\ref{fig:results_distance_all_dis} and Figure~\ref{fig:results_time_all_dis} show that the best scheduler-route optimizer combination was EA-ACO, minimizing the distance travelled and time taken to travel to as less as 150 km and 0.2 hours, as compared to almost 500 kms and 1 hour for the worst combination. 
It can be observed that ACO is the top 4 best route optimizers, whereas the least optimal performance was showed EA nad PSO route optimizers, and MILP and simulated annealing schedulers.


\subsection{Phase 2 and 3: Experiment Results}
The experiments were conducted by combining different scheduling algorithms with different route optimization algorithms. The findings of each graph is summarized below.



The best performing scheduler-route optimizer combinations for each distributor are as follows:

\begin{table}[H]
    \centering
    \resizebox{\textwidth}{!}{%
    \begin{tabular}{|l|l|l|}
        \hline
        \textbf{Distributor} & \textbf{Best Distance Minimized Scheduler-Route Optimizer} & \textbf{Best Travel Time Minimized Scheduler-Route Optimizer} \\
        \hline
        Distributor 7 & EA-ACO & EA-ACO \\
        Distributor 6 & EA-ACO & EA-ACO \\
        Distributor 1 & EA-ACO & EA-ACO \\
        Distributor 131 & ACO-EA & ACO-EA \\
        \hline
    \end{tabular}%
    }
    \caption{Best performing Scheduler-Route Optimizer combinations by distributor}
    \label{tab:best_schedulers}
    \end{table}
    

\subsection{Conclusion}
Based on the results from the experiments, the combination of Evolutionary Algorithm (EA) for scheduling and Ant Colony Optimization (ACO) for route optimization proved to be the best-performing solution, as evidenced by its consistent performance across all distributors, particularly in minimizing both distance and travel time.

The ACO route optimizer, in particular, outperformed all other route optimization algorithms across the experiments. For distributor 131, however, the ACO-EA combination was identified as the best solution. This highlights the importance of considering both the scheduling and route optimization algorithms in tandem when evaluating their overall performance in real-world scenarios.

